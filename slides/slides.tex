%
% ---------------------------------------------------------------
% Copyright (C) 2012-2018 Gang Li
% ---------------------------------------------------------------
%
% This work is the default powerdot-tuliplab style test file and may be
% distributed and/or modified under the conditions of the LaTeX Project Public
% License, either version 1.3 of this license or (at your option) any later
% version. The latest version of this license is in
% http://www.latex-project.org/lppl.txt and version 1.3 or later is part of all
% distributions of LaTeX version 2003/12/01 or later.
%
% This work has the LPPL maintenance status "maintained".
%
% This Current Maintainer of this work is Gang Li.
%
%

\documentclass[
 size=14pt,
 paper=smartboard,  %a4paper, smartboard, screen
 mode=present, 		%present, handout, print
 display=slides, 	% slidesnotes, notes, slides
 style=tuliplab,  	% TULIP Lab style
 pauseslide,
 fleqn,leqno]{powerdot}


\usepackage{cancel}
\usepackage{caption}
\usepackage{stackengine}
\usepackage{smartdiagram}
\usepackage{attrib}
\usepackage{amssymb}
\usepackage{amsmath} 
\usepackage{amsthm} 
\usepackage{mathtools}
\usepackage{rotating}
\usepackage{graphicx}
\usepackage{boxedminipage}
\usepackage{rotate}
\usepackage{calc}
\usepackage[absolute]{textpos}
\usepackage{psfrag,overpic}
\usepackage{fouriernc}
\usepackage{pstricks,pst-3d,pst-grad,pstricks-add,pst-text,pst-node,pst-tree}
\usepackage{moreverb,epsfig,subfigure}
\usepackage{color}
\usepackage{booktabs}
\usepackage{etex}
\usepackage{breqn}
\usepackage{multirow}
\usepackage{natbib}
\usepackage{bibentry}
\usepackage{gitinfo2}
\usepackage{siunitx}
\usepackage{nicefrac}
%\usepackage{geometry}
%\geometry{verbose,letterpaper}
\usepackage{media9}
\usepackage{animate}
%\usepackage{movie15}
\usepackage{auto-pst-pdf}

\usepackage{breakurl}
\usepackage{fontawesome}
\usepackage{xcolor}
\usepackage{multicol}



\usepackage{verbatim}
\usepackage[utf8]{inputenc}
\usepackage{dtk-logos}
\usepackage{tikz}
\usepackage{adigraph}
%\usepackage{tkz-graph}
\usepackage{hyperref}
%\usepackage{ulem}
\usepackage{pgfplots}
\usepackage{verbatim}
\usepackage{fontawesome}


\usepackage{todonotes}
% \usepackage{pst-rel-points}
\usepackage{animate}
\usepackage{fontawesome}

\usepackage{listings}
\lstset{frameround=fttt,
frame=trBL,
stringstyle=\ttfamily,
backgroundcolor=\color{yellow!20},
basicstyle=\footnotesize\ttfamily}
\lstnewenvironment{code}{
\lstset{frame=single,escapeinside=`',
backgroundcolor=\color{yellow!20},
basicstyle=\footnotesize\ttfamily}
}{}


\usepackage{hyperref}
\hypersetup{ % TODO: PDF meta Data
  pdftitle={Presentation Title},
  pdfauthor={Gang Li},
  pdfpagemode={FullScreen},
  pdfborder={0 0 0}
}


% \usepackage{auto-pst-pdf}
% package to show source code

\definecolor{LightGray}{rgb}{0.9,0.9,0.9}
\newlength{\pixel}\setlength\pixel{0.000714285714\slidewidth}
\setlength{\TPHorizModule}{\slidewidth}
\setlength{\TPVertModule}{\slideheight}
\newcommand\highlight[1]{\fbox{#1}}
\newcommand\icite[1]{{\footnotesize [#1]}}

\newcommand\twotonebox[2]{\fcolorbox{pdcolor2}{pdcolor2}
{#1\vphantom{#2}}\fcolorbox{pdcolor2}{white}{#2\vphantom{#1}}}
\newcommand\twotoneboxo[2]{\fcolorbox{pdcolor2}{pdcolor2}
{#1}\fcolorbox{pdcolor2}{white}{#2}}
\newcommand\vpspace[1]{\vphantom{\vspace{#1}}}
\newcommand\hpspace[1]{\hphantom{\hspace{#1}}}
\newcommand\COMMENT[1]{}

\newcommand\placepos[3]{\hbox to\z@{\kern#1
        \raisebox{-#2}[\z@][\z@]{#3}\hss}\ignorespaces}

\renewcommand{\baselinestretch}{1.2}


\newcommand{\draftnote}[3]{
	\todo[author=#2,color=#1!30,size=\footnotesize]{\textsf{#3}}	}
% TODO: add yourself here:
%
\newcommand{\gangli}[1]{\draftnote{blue}{GLi:}{#1}}
\newcommand{\shaoni}[1]{\draftnote{green}{sn:}{#1}}
\newcommand{\gliMarker}
	{\todo[author=GLi,size=\tiny,inline,color=blue!40]
	{Gang Li has worked up to here.}}
\newcommand{\snMarker}
	{\todo[author=Sn,size=\tiny,inline,color=green!40]
	{Shaoni has worked up to here.}}

%%%%%%%%%%%%%%%%%%%%%%%%%%%%%%%%%%%%%%%%%%%%%%%%%%%%%%%%%%%%%%%%%%%%%%%%
% title
% TODO: Customize to your Own Title, Name, Address
%
\title{New York City Taxi Fare Prediction }
\author{
Zhenshuai Xu
\\
\\JiLin University
\\Deakin University
\\Chinese Academy of Sciences
}
\date{\gitCommitterDate}


% Customize the setting of slides
\pdsetup{
% TODO: Customize the left footer, and right footer
rf=\href{http://www.tulip.org.au}{
Last Changed by: \textsc{\gitCommitterName}\ \gitVtagn-\gitAbbrevHash\ (\gitAuthorDate)
},
cf={New York City Taxi Fare Prediction},
}


\begin{document}

\maketitle

%\begin{slide}{Overview}
%\tableofcontents[content=sections]
%\end{slide}


%%==========================================================================================
%%
\begin{slide}[toc=,bm=]{Overview}
	
\tableofcontents[content=currentsection,type=1]
\end{slide}
%%
%%==========================================================================================


\section{Problem Definition}


%%==========================================================================================
%%
\begin{slide}{New York City Taxi Fare Prediction}
\begin{center}
\twotonebox{\rotatebox{90}{Defn}}{\parbox{.86\textwidth}
{The goal of this project is to predict the fare amount of a taxi ride given the input features like pickup_datetime, pickup/dropoff latitude, pickup/dropoff longitude and number of passengers. This is a Supervised regression machine learning task. 
\begin{itemize}
\item A passenger may be interested in the \textcolor{orange}{taxi fare} that
differ between \textcolor{orange}{different regions} .
\item Driver would prefer to how can them make more money.
\end{itemize}
}}

\end{center}

\end{slide}
%%==========================================================================================

\section{Data Clean}
%%==========================================================================================


%%
%%==========================================================================================


%%==========================================================================================
%%
\subsection{Data Analysis}
\begin{slide}[toc=,bm=]{Data Describe}
\begin{center}
	\begin{figure}[htbp]
		
		\includegraphics{./figures/1.eps}
		\caption{ Describe}
	\end{figure}

\end{center}

\bigskip %表示垂直间距



\end{slide}
%%



%%==========================================================================================
\subsection{Data Visualization}
\begin{slide}[toc=,bm=]{Data Visualization Box Plot}
	\begin{center}
		\begin{figure}[htbp]
			
			\includegraphics[scale=0.5]{./figure/1.eps}
			\qquad
			\includegraphics[scale=0.5]{./figure/2.eps}
			\caption{box plot }
		\end{figure}
		
	\end{center}
	
	\bigskip %表示垂直间距
	
	
	
\end{slide}


%%
%%==========================================================================================



\begin{slide}[toc=,bm=]{Data Visualization Scatter Plot}
	
	\begin{figure}[htbp]
		\centering
		\includegraphics[scale=0.5]{./figure/3.eps}
		%	\caption{pick up longtitude }		
		\qquad
		\includegraphics[scale=0.5]{./figure/4.eps}
		%	\caption{passenger count}
		\includegraphics[scale=0.5]{./figure/5.eps}
		\caption{scatter plot to find Outlier}
	\end{figure}
	
	
	\bigskip %表示垂直间距
	
	
	
\end{slide}

\subsection{Data Clean}
\begin{slide}[toc=,bm=]{Data Clean}
	
	\begin{figure}[htbp]
		\centering
		\includegraphics[scale=0.5]{./figure/6.eps}
		%	\caption{pick up longtitude }		
		\qquad
		\includegraphics[scale=0.5]{./figure/7.eps}
		%	\caption{passenger count}
		\includegraphics[scale=0.5]{./figure/8.eps}
		\caption{scatter plot to find Outlier}
	\end{figure}
	
	
	\bigskip %表示垂直间距
	
	
	
\end{slide}

\subsection{Further Visualization of Data}
\begin{slide}[toc=,bm=]{Further Visualization of Data}

	\begin{figure}[htbp]
		\centering
		
		\includegraphics[scale=0.2]{./figure/15.eps}
		%	\caption{pick up longtitude }		
		\qquad
		\includegraphics[scale=0.2]{./figure/16.eps}
		%	\caption{passenger count}
		\qquad
		\includegraphics[scale=0.48]{./figure/17.eps}
		\caption{Location Visualization}
	\end{figure}
	
	
	\bigskip %表示垂直间距
	
	
	
\end{slide}
%%==========================================================================================
%%

\begin{slide}[toc=,bm=]{Further Visualization of Data}
	
	\begin{figure}[htbp]
		\centering
		
		\includegraphics[scale=0.5]{./figure/11.eps}
		%	\caption{pick up longtitude }		
		\caption{Map Visualization}
	\end{figure}
	
	
	\bigskip %表示垂直间距
	
	
	
\end{slide}

\begin{slide}[toc=,bm=]{Further Visualization of Data}
	
	\begin{figure}[htbp]
		\centering
		
		\includegraphics[scale=0.5]{./figure/13.eps}
		%	\caption{pick up longtitude }		
		\qquad
		\includegraphics[scale=0.4]{./figure/12.eps}
		%	\caption{passenger count}
		\includegraphics[scale=0.5]{./figure/14.eps}
		\caption{ Visualization to Remove Water Points}
	\end{figure}
	
	
	\bigskip %表示垂直间距
	
	
\end{slide}	


\section{Knowledge Discovery}
\subsection{Regional Relationship Discovery}
\begin{slide}[toc=,bm=]{Regional Relationship Discovery}
	
	\begin{figure}[htbp]
		\centering
		
		\includegraphics[scale=0.5]{./figure/18.eps}
		%	\caption{pick up longtitude }		
		\caption{Cluster Analysis}
	\end{figure}
	
	
	\bigskip %表示垂直间距
	
	
	
\end{slide}
\subsection{Variable Relationship Discovery}

\begin{slide}[toc=,bm=]{Variable Relationship Discovery}
	
	\begin{figure}[htbp]
		\centering
		
		\includegraphics[scale=0.5]{./figure/20.eps}
		%	\caption{pick up longtitude }		
		\caption{Heat Map}
	\end{figure}
	
	
	\bigskip %表示垂直间距
	
	
	
\end{slide}

\begin{slide}[toc=,bm=]{Variable Relationship Discovery}
	
	\begin{figure}[htbp]
		\centering
		
		\includegraphics[scale=0.5]{./figure/9.eps}
		%	\caption{pick up longtitude }
		\qquad	
		\includegraphics[scale=0.5]{./figure/19.eps}	
		\caption{Cluster Analysis}
	\end{figure}
	
	
	\bigskip %表示垂直间距
	
	
	
\end{slide}

\section{Feature Engineering}
\subsection{Haversine Distance}
\begin{slide}[toc=,bm=]{Haversine Distance}
	
	\begin{figure}[htbp]
		\centering
		
		\includegraphics[scale=0.5]{./figure/21.eps}
		%	\caption{pick up longtitude }
		\qquad
			\includegraphics[scale=0.5]{./figure/22.eps}		
		\caption{Haversine Distance}
	\end{figure}
	
	
	\bigskip %表示垂直间距
	
	
	
\end{slide}

\subsection{Distance to Center}
\begin{slide}[toc=,bm=]{Distance to Center}
	
	\begin{figure}[htbp]
		\centering
		
		\includegraphics[scale=0.5]{./figure/23.eps}		
		\caption{Distance to NYC}
	\end{figure}
	
	
	\bigskip %表示垂直间距
	
	
	
\end{slide}

\begin{slide}{Group Outlying Aspects Mining}
\twotonebox {\rotatebox{90}{Defn}}{\parbox{.88\textwidth}
{
{\textcolor{orange}{Group outlying aspects mining} aims to
identify the outstanding features of the group of query object.
\begin{itemize}
\item
Doctors desire to identify the merits \& demerits between
\textcolor{orange}{a group of cancer patients} and normal people.
\item
NBA coaches are passionate about exploring the obvious advantages \&
disadvantages of \textcolor{orange}{the team}.
\end{itemize}
}
}}

\vspace{1.5cm}

\twocolumn{
\begin{figure}
  \centering
  \selectcolormodel{rgb}
  \missingfigure{Testing.}
  %\includegraphics[width=0.6\textwidth]{figures//demical.eps}\\
  \caption{Medical}\label{fig:demical}
\end{figure}
}{
\begin{figure}
  \centering
  \selectcolormodel{rgb}
  \missingfigure{Testing.}
  %\includegraphics[width=0.6\textwidth]{figures//NBA_team.eps}\\
  \caption{NBA-Team}\label{fig:timg}
\end{figure}
}



%%==========================================================================================
\begin{note}
However,
there is such a phenomenon in real life.
Doctors desire to identify the characteristics between
a group of cancer patients and normal people.
NBA coaches are passionate about exploring the obvious strengths and
weaknesses of the team compared with other teams.

Based on such a phenomenon in the real life,
we proposed the concept of group outlying aspects mining.
\end{note}
%%==========================================================================================

\end{slide}
%%
%%==========================================================================================


%%==========================================================================================
%%
\begin{slide}[toc=,bm=]{Problem Formalization}
\twotonebox {\rotatebox{90}{Defn}}{\parbox{.88\textwidth}
{
{\textcolor{orange}{Group outlying aspects mining} aims to identify
the \textcolor{orange}{top-k group outlying subspace $s \subseteq F$} in
which the query group $G_q$ is \textcolor{orange} {distinctive with other groups}.
\begin{itemize}
\item
$G = \{G_q, G_2, G_3,..., G_n\}$ $\Leftrightarrow$ a set of groups.
\item
$G_q$ $\Leftrightarrow$ the query group.
\item
Other groups $\Leftrightarrow$ comparison groups.
\item
Each object in the group has $d$ features $F = \{f_1, f_2, ..., f_d\}$.
\end{itemize}
}
}}

%%==========================================================================================
\begin{note}
Next,
let me talk about the concept of group outlying aspects mining.

For example,
Dataset $G$ has $n$ groups.
$G_q$ is the query group.
and other groups are comparison groups.
Each object in the group has d features $F = $ $f_1$, $f_2$, $f_3$ to $f_d$.
The group outlying aspects mining is to identify the top-k group outlying subspaces,
which are different from other groups.

What does the top-k group outlying subspaces mean?
Next, I will explain it.
\end{note}

%%==========================================================================================
\end{slide}
%%
%%==========================================================================================


%%==========================================================================================
%%
\begin{slide}[toc=,bm=]{Term Definition}
\begin{itemize}
\item
Top-k group outlying subspaces

\begin{itemize}
\item
$\rho_s(\cdot)$ $\Rightarrow$ outlying scoring function.

\item
$\rho_s(\cdot)$ quantifies the outlying degree of the
query group $G_q$ in the subspace $s$.

\item
Order by DESC using scoring function $\rho(\cdot)$
to identify top K group outlying subspaces.
\end{itemize}
\end{itemize}

\begin{figure}[htbp]
    \centering
    \subfigure[Original Feature Spaces]{
      \selectcolormodel{rgb}
      \missingfigure[figwidth=5.5cm]{Test.}
        %\includegraphics[width=0.3\textwidth]{figures//example-basketball-original.eps}
        \label{fig:basketball-original}
    }
    \subfigure[Group Outlying Spaces]{
       \selectcolormodel{rgb}
       \missingfigure[figwidth=5.5cm]{Test.}
        %\includegraphics[width=0.3\textwidth]{figures//example-basketball-projection.eps}
        \label{fig:basketball-projection}
    }
    \subfigure[Another Subspaces]{
      \selectcolormodel{rgb}
      \missingfigure[figwidth=5.5cm]{Test.}
        %\includegraphics[width=0.28\textwidth]{figures//basketball-another-subspaces.eps}
        \label{fig:basketball-projection1}
    }
%    \caption{Histogram representation of a group on three single features}
    \label{fig:Basketball-Example}
\end{figure}

%%==========================================================================================
\begin{note}
We use $\rho_s(\cdot)$ to describe an outlying scoring function.
$\rho_s(\cdot)$ quantifies the outlying degree of the query group in a subspaces $s$.
In addition,
we use the scoring function to make a descending order and at last
filter out the top k group outlying subspaces.
It is obvious that the outlying subspaces make the
query group different from other groups.
\end{note}
%%==========================================================================================

\end{slide}
%%
%%==========================================================================================


%%==========================================================================================
%%
\begin{slide}[toc=,bm=]{Term Definition}
\begin{itemize}
\item
Trivial Outlying Features

\begin{itemize}
\item
\smallskip
One-dimension subspaces.

\item
${G_q}$'s outlying degree $\rho(\cdot)$ $>$ $\alpha$.
\end{itemize}
\end{itemize}
\begin{table}
\setlength{\abovecaptionskip}{0pt}
\setlength{\belowcaptionskip}{10pt}
\centering
\caption{$\alpha = 4$}

\begin{tabular}{  c  |  c }
\toprule
\centering
\texttt{Feature}  & \texttt{Outlying Degree}  \\
\midrule
 {\textcolor{orange}{\{$F_1$\}}} & $4.351$ \\
 {\{$F_3, F_4$\}}                & $4.024$ \\
 {\{$F_2, F_4$\}}                & $2.318$ \\
 {\{$F_2$\}}                     & $2.002$ \\
 {\{$F_3$\}}                     & $1.028$ \\
\bottomrule
\end{tabular}
\end{table}

%%==========================================================================================
\begin{note}
In order to identify the top-k outlying subspaces,
we categorize the features into $2$ non-overlapping groups,
trivial outlying features and non-trivial outlying subspaces.

First, let me introduce the trivial outlying features.

Trivial outlying features are one-dimension subspaces.
In the subspace,
the query group's outlying degree is larger than the threshold $\alpha$.

We can see from table $1$,
when the specified threshold $\alpha = 4$,
the trivial outlying feature is \{$F_1$\}.
\end{note}
%%==========================================================================================

\end{slide}
%%
%%==========================================================================================


%%==========================================================================================
%%
\begin{slide}[toc=,bm=]{Term Definition}
\begin{itemize}
\item
Non-Trivial Outlying Subspaces
\begin{itemize}
\item
\smallskip
Multi-dimension subspaces.

\item
\smallskip
${G_q}$'s outlying degree $\rho(\cdot)$ $>$ $\alpha$.
\end{itemize}
\end{itemize}

\begin{table}
\setlength{\abovecaptionskip}{0pt}
\setlength{\belowcaptionskip}{10pt}
\centering
\caption{$\alpha = 4$}

\begin{tabular}{  c  |  c }
\toprule
\centering
\texttt{Feature}  & \texttt{Outlying Degree}  \\
\midrule
 {\{$F_1$\}}                           & $4.351$ \\
 {\textcolor{orange}{\{$F_3, F_4$\}}}  & $4.024$ \\
 {\{$F_2, F_4$\}}                      & $2.318$ \\
 {\{$F_2$\}}                           & $2.002$ \\
 {\{$F_3$\}}                           & $1.028$ \\
\bottomrule
\end{tabular}
\end{table}

%%==========================================================================================
\begin{note}
Next,
I will introduce the non-trivial outlying subspaces.
Non-Trivial outlying subspaces are multi-dimension subspaces.
In the subspace,
the query group's outlying degree is larger than the threshold $\alpha$.

Table $2$ shows that,
when the threshold $\alpha$ equal four,
the non-trivial outlying subspace is \{$F_3$, $F_4$\}.
\end{note}
%%==========================================================================================

\end{slide}
%%
%%==========================================================================================


\section{Related Work and Challenges}


%%==========================================================================================
%%
\begin{slide}{Related Work - Outlying Aspects Mining}
%Related Work - Outlying Aspects Mining
\begin{itemize}
\item
Existing Methods - \textcolor{orange}{Feature selection}

\begin{itemize}
\item
To distinguish two classes:
the query point (positive) \& rest of data (negative)
\end{itemize}
\vspace{1cm}
\twocolumn[
\savevalue{lfrheight}=5cm,
\savevalue{lfrprop}={
linestyle=solid,framearc=.2,linewidth=1pt},
rfrheight=\usevalue{lfrheight},
rfrprop=\usevalue{lfrprop}
]{
Disadvantages
\begin{itemize}
\item
\smallskip
Positive and negative classes are \textcolor{orange}{Not} balanced.

\item
\smallskip
\textcolor{orange}{Not} quantify the outlying degree accurately.

\item
\smallskip
\textcolor{orange}{Not} identify group outlying aspects.
\end{itemize}
}
{
Advantages
\begin{itemize}
\item
\smallskip
Easy to operate.

\item
\smallskip
Resolve dimensionality bias.
\end{itemize}
}
\end{itemize}

%%==========================================================================================
\begin{note}
Let me introduce two existing methods:
Feature selection and score-and-search.

For feature selection,
the query point can be regarded as positive class and
the rest of the data can be regarded as negative class,
selected the features that best distinguish the two classes.

The advantages of this method are easy to operate,
and it's able to resolve dimensionality bias.
However, it has some drawbacks.
Firstly,
positive and negative classes are Not balanced,
secondly,
it can't quantify the outlying degree correctly.
Most importantly,
it doesn't identify group outlying aspects.
\end{note}
%%==========================================================================================

\end{slide}
%%
%%==========================================================================================


%%==========================================================================================
%%
\begin{slide}[toc=,bm=]{Related Work - Outlying Aspects Mining}

\begin{itemize}
\item
Existing Methods - \textcolor{orange} {Score-and-search}

\begin{itemize}
\item
Define an outlying score function.

\item
Search subspaces.
\end{itemize}
\bigskip
\twocolumn[
\savevalue{lfrheight}=5cm,
\savevalue{lfrprop}={
linestyle=solid,framearc=.2,linewidth=1pt},
rfrheight=\usevalue{lfrheight},
rfrprop=\usevalue{lfrprop}
]{
Disadvantages
\begin{itemize}
\item
\smallskip
Dimensionality bias.

\item
\smallskip
Search efficiency is \textcolor{orange}{Not} high (dataset is large).

\item
\smallskip
\textcolor{orange}{Not} identify group outlying aspects.
\end{itemize}
}{
Advantages
\begin{itemize}
\item
\smallskip
Quantify the outlying degree correctly.

\item
\smallskip
High Comprehensibility.

\end{itemize}
}
\end{itemize}

%%==========================================================================================
\begin{note}
For score-and-search method,
it defines an outlying score function,
and then searches for each subspace until it finds out a subspace that
makes the query point show the best score.

The advantages of this method are:
first, it enables to quantify the outlying degree accurately.
Besides, its comprehensibility is high.

While the disadvantages include three main aspects:
the first one is it has dimensionality bias;
Additionally, its search efficiency is low.
Last but not least, it doesn't identify group outlying aspects.
\end{note}
%%==========================================================================================

\end{slide}
%%
%%==========================================================================================


%%==========================================================================================
%%
\begin{slide}[toc=,bm=]{}
\twocolumn
{
Group Outlying Aspects Mining
\begin{itemize}
\item
\smallskip
Focus on differences between \textcolor{orange}{groups}.

\item
\smallskip
\textcolor{orange}{Multiple} points.
\medskip
\end{itemize}
\vspace{0.75cm}
%\vspace{0.1cm}
\begin{figure}
  \centering
  \selectcolormodel{rgb}
  \missingfigure{Testing a long text string.}
  %\includegraphics[width=0.6\textwidth]{figures//example-basketball-projection.eps}\\
  \caption{Group Outlying Aspects Target}\label{fig:GroupOutAspect-target}
\end{figure}
}
{
Outlying Aspects Mining
\begin{itemize}
\item
Concentrates on differences between \textcolor{orange}{objects}.

\item
\textcolor{orange}{One} point.
\end{itemize}
\bigskip
\begin{figure}
  \centering
  \selectcolormodel{rgb}
  \missingfigure{Testing a long text string.}
%  \includegraphics[width=0.5\textwidth]{figures//OutAspect_target.eps}\\
  \caption{Outlying Aspects Target}\label{fig:OutAspect-target}
\end{figure}
}

%%==========================================================================================
\begin{note}
In this research paper,
we proposed the group outlying aspects mining.
Now,
let me summarize the differences between group outlying aspects mining and outlying aspects mining.

Group outlying aspects mining mainly focuses on the differences between groups.
But outlying aspects mining mainly concentrates on the differences between objects.
The target of group outlying aspects mining can be seen as many points.
While the target of outlying aspects mining can be regarded as one point.

In the NBA example,
group outlying aspects mining focuses on the advantages
or disadvantages of one team,
however,
outlying aspects mining focuses on the advantages or disadvantages of one player.
\end{note}
%%==========================================================================================

\end{slide}
%%
%%==========================================================================================


%%==========================================================================================
%%
\begin{slide}{Challenges (1)}
%Challenges (1)
\begin{itemize}
\item
How to \textcolor{orange}{represent} the group features.

\begin{itemize}
\item
Can be affected by outlier values.

\item
Can \textcolor{orange}{Not} reflect the overall distribution of group features.
\end{itemize}
\end{itemize}

%%==========================================================================================
\begin{note}
Based on current existing methods,
there still remains some research challenges.

The first one is how to represent the group features
based on the features of the individuals in the group.

Although the arithmetic mean of all elements
in each feature can describe the features of one group.
It can be affected by outlier values,
and can't reflect the entire distribution of group features.L
\end{note}
%%==========================================================================================

\end{slide}
%%
%%==========================================================================================


%%==========================================================================================
%%
\begin{slide}[toc=,bm=]{Challenges (2)}

\begin{itemize}
\item
How to \textcolor{orange}{evaluate} the outlying degree in different aspects.

\begin{itemize}
\item
Need design a scoring function when necessary.

\item
Adopting an appropriate scoring function (without dimension bias) remains a problem.

\end{itemize}
\end{itemize}

%%==========================================================================================
\begin{note}
The second challenge is how to evaluate the outlying degree of
the query group between different aspects.

In that case,
we need to design a scoring function to measure the outlying degree.
But adopting an appropriate scoring function without dimension bias still remains a problem.
\end{note}
%%==========================================================================================

\end{slide}
%%
%%==========================================================================================


%%==========================================================================================
%%
\begin{slide}[toc=,bm=]{Challenges (3)}

\begin{itemize}
\item
How to \textcolor{orange}{improve} the efficiency.

\begin{itemize}

\item
When the dimension of the \textcolor{orange}{data is high},
the candidate subspace grows exponentially.

\item
It will easily go beyond the limits of the computation resources.

\end{itemize}
\end{itemize}

%%==========================================================================================
\begin{note}
The third challenge is how to improve efficiency.

To be specific,
when the dimension of data is high,
the candidate subspace increase dramatically,
so that it is very easy to exceed the limit of computer resources.
\end{note}
%%==========================================================================================

\end{slide}
%%
%%==========================================================================================


\section{GOAM Algorithm}


%%==========================================================================================
%%
\begin{slide}[toc=,bm=]{}

Framework of GOAM algorithm:

\bigskip

\begin{figure}
  \centering
  \selectcolormodel{rgb}
  \missingfigure{Testing a long text string.}
%  \includegraphics[width=0.55\textwidth]{figures//framework1.eps}\\
  \caption{Framework of GOAM Algorithm} \label{framework}
\end{figure}

%%==========================================================================================
\begin{note}
In order to tackle the above issues,
GOAM algorithm is involved.

Let's have a look at the framework of this algorithm.
The first step is to use the histogram to represent the group features
based on all individuals in the group.

Following that,
we utilize the earth mover distances to measure the
outlying degree between groups.
This is the second step:
outlying degree scoring.

The last step is to identify the outlying aspects.

\end{note}
%%==========================================================================================

\end{slide}
%%
%%==========================================================================================


%%==========================================================================================
%%
\begin{slide}{Step One - Group Feature Extraction}
%Step One - Group Feature Extraction}
\begin{itemize}
\item
\smallskip
Suppose $f_1$, $f_2$, $f_3$ are three features of $G_q$.

$f_1$: \{$x_1, x_2, x_3, x_4, x_5, x_2, x_3, x_4, x_1, x_2$\} \\

$f_2$: \{$y_2, y_2, y_1, y_2, y_3, y_3, y_5, y_4, y_4, y_2$\} \\

$f_3$: \{$z_1, z_4, z_2, z_4, z_5, z_3, z_1, z_2, z_4, z_2$\} \\
\end{itemize}

\begin{figure}[htbp]
    \centering
    \subfigure[$f_1$]{
        \selectcolormodel{rgb}
        \missingfigure[figwidth=5.5cm]{Test.}
        %\includegraphics[width=0.25\textwidth]{figures//frequency-distribution-feature1.eps}
        \label{fig:fre-dis-f1}
    }
    \subfigure[$f_2$]{
        \selectcolormodel{rgb}
        \missingfigure[figwidth=5.5cm]{Test.}
        \label{fig:fre-dis-f2}
    }
    \subfigure[$f_3$]{
        \selectcolormodel{rgb}
        \missingfigure[figwidth=5.5cm]{Test.}
        \label{fig:fre-dis-f3}
    }
    \caption{Histogram of $G_q$ on three features}
    \label{fig:fre-dis-each-feature}
\end{figure}

%%==========================================================================================
\begin{note}
Now, let me specifically explain what each step means.
The first step is group feature extraction.
we can take one group extraction as an example.

We suppose to use $f_1$, $f_2$, $f_3$ to represent three features of $G_q$.
The values of $f_1$ are {$x_1$, $x_2$, $x_3$, $x_4$} and so on.
And the values of $f_2$ are {$y_2$, $y_2$, $y_1$, $x_2$} and so on.

For feature $f_1$,
we use the histogram to illustrate feature $f_1$ after
counting the frequency of each value,
as show in figure 6 (a).

Similarly,
we can also extract other features of the group
according to feature $f_1$.
\end{note}
%%==========================================================================================

\end{slide}
%%
%%==========================================================================================


%%==========================================================================================
%%
\begin{slide}{Step Two - Outlying Degree Scoring}
%Step Two - Outlying Degree Scoring
\begin{itemize}
\item
Calculate Earth Mover Distance

\begin{itemize}
\item
Represent one feature among different groups

\item
Purpose: calculate the minimum mean distance
\end{itemize}

\begin{figure}
   \selectcolormodel{rgb}
   \missingfigure{Make a sketch of the structure of a trebuchet.}
%  \includegraphics[width=0.4\textwidth]{figures//example3.eps}\\
   \caption{EMD of one feature}\label{EMD}
\end{figure}
\end{itemize}

%%==========================================================================================
\begin{note}
The second step is outlying degree scoring,
which is to evaluate the outlying degree between the target group and competitive groups.

First,
we calculate the earth mover distance of one feature in different groups.

The earth mover distance reflects the minimum mean distance between
the target group and other groups on one feature.

Later on,
we utilize the EMD to measure the differences between groups.
\end{note}
%%==========================================================================================

\end{slide}
%%
%%==========================================================================================


%%==========================================================================================
%%
\begin{slide}[toc=,bm=]{Step Two - Outlying Degree Scoring}

\begin{itemize}
\item
Calculate the outlying degree

\vspace{1.2cm}

\begin{centering}

$ OD(G_q) = \sum_{1}^{n}EDM(h_{q_s}, h_{k_s}) $

\end{centering}

\begin{itemize}
\item
n $\Leftrightarrow$ the number of contrast groups.

\item
$h_{k_s}$  $\Leftrightarrow$ the histogram representation of $G_k$ in the subspace s.

\end{itemize}
\end{itemize}

%%==========================================================================================
\begin{note}
Base on the earth mover distance,
we can calculate the outlying degree using the formula shown on the screen.

This formula is the outlying degree scoring function,
n represents the number of competitive groups.
$h_{k_s}$ is the histogram of $G_k$ in the subspace s.
\end{note}
%%==========================================================================================

\end{slide}
%%
%%==========================================================================================


%%==========================================================================================
%%
\begin{slide}{Step Three - Outlying Aspects Identification}
%Step Three - Outlying Aspects Identification
\begin{itemize}
\item
Identify group outlying aspects mining based on the value
of outlying degree.

\item
The greater the outlying degree is,
the more likely it is group outlying aspect.
\end{itemize}

%%==========================================================================================
\begin{note}
Next,
let me talk about the third step.

In this step,
we identify the group outlying aspects according to the value of the outlying degree.

If a feature's outlying degree is greater,
it is more likely to be a group outlying aspect.
\end{note}
%%==========================================================================================

\end{slide}
%%
%%==========================================================================================


%%==========================================================================================
%%
\begin{slide}[toc=,bm=]{Pseudo code}

\begin{itemize}
\item
Pseudo code of GOAM algorithm

\end{itemize}

\begin{figure}
  \centering
  \selectcolormodel{rgb}
  \missingfigure[figwidth=16cm]{Testing a long text string}
  %\includegraphics[width=0.75\textwidth]{figures//GOAM.eps}\\
  %\caption{GOAM Algorithm}\label{OS-Identification}
\end{figure}

%%==========================================================================================
\begin{note}
Last,
we use the outlying degree to identify the specific group outlying aspects.

The pseudo code of GOAM algorithm is as follows.
The input is the group data,
the output is outlying aspects of specific group ($G_1$).

The details of the algorithm I will use an example to explain.
\end{note}

%%==========================================================================================
\end{slide}
%%
%%==========================================================================================


%%==========================================================================================
%%
\begin{slide}[toc=,bm=]{Illustration}
% Outlying Aspects Identification
\begin{table}
\setlength{\abovecaptionskip}{0pt}
\setlength{\belowcaptionskip}{10pt}
\centering
\caption{Original Dataset}
\begin{tabular}{ccccc | ccccc}
  \toprule
  $G_1$ & $F_1$ & $F_2$ & $F_3$ & $F_4$ & $G_2$ & $F_1$ & $F_2$ & $F_3$ & $F_4$ \\
  \midrule
   &10 & 8 & 9 & 8 & &7 & 7 & 6 & 6 \\
   &9  & 9 & 7 & 9 & &8 & 9 & 9 & 8 \\
   &8  & 10& 8 & 8 & &6 & 7 & 8 & 9  \\
   &8  & 8 & 6 & 7 & &7 & 7 & 7 & 8  \\
   &9  & 9 & 9 & 8 & &8 & 6 & 6 & 7  \\
   \midrule
   $G_3$ & $F_1$ & $F_2$ & $F_3$ & $F_4$ & $G_4$ & $F_1$ & $F_2$ & $F_3$ & $F_4$ \\
   \midrule
   &8 & 10 & 8 & 8 & &9 & 8 & 8 & 8\\
   &9 & 9  & 7 & 9 & &7 & 7 & 7 & 9\\
   &10& 9  & 10& 7 & &8 & 6 & 6 & 8\\
   &9 & 10 & 8 & 6 & &9 & 8 & 8 & 7\\
   &9 & 9  & 7 & 9 & &8 & 7 & 9 & 8\\
  \bottomrule
\end{tabular}
\end{table}

%%==========================================================================================
\begin{note}
Next,
let me use an example to explain GOAM algorithm.

Suppose we have four groups.
Each group has four features,
and
the specific values are shown in table $3$.
\end{note}
%%==========================================================================================

\end{slide}
%%
%%==========================================================================================


%%==========================================================================================
%%
\begin{slide}[toc=,bm=]{Illustration}

\setlength{\abovecaptionskip}{0pt}
\setlength{\belowcaptionskip}{10pt}
\centering
\begin{table}
\caption{outlying degree of each possible subspaces}

\begin{tabular}{c|c|c|c}
  \toprule
  % after \\: \hline or \cline{col1-col2} \cline{col3-col4}
  Feature & Outlying Degree & Feature & Outlying Degree \\
  \midrule
  \textcolor{orange}{\{$F_1$\}}  & 4.351  & \{$F_2, F_3$\}  & 4.023 \\
  \{$F_2$\}  & 2.012                      & \textcolor{orange}{\{$F_3, F_4$\}} & 4.324 \\
  \{$F_3$\}  & 1.392                      & \{$F_2, F_4$\} & 2.018 \\
  \{$F_4$\}  & 2.207                      & \{$F_2, F_3, F_4$\} & 2.012 \\
  \bottomrule
\end{tabular}
\end{table}

\bigskip

\twocolumn{
\begin{itemize}
\item
Search process:\\
$OD(\{$$F_1$$\}) > \alpha$, save to $T_1$.\\
$OD(\{$$F_2$$\}) < \alpha$, save to $C_1$.\\
$OD(\{$$F_3$$\}) < \alpha$, save to $C_2$. \\
$OD(\{$$F_4$$\}) < \alpha$, save to $C_3$. \\
\end{itemize}}
{
\vspace{.75cm}
$OD(\{$$F_2, F_3$$\}) > \alpha$, save to $N_1$. \\
$OD(\{$$F_3, F_4$$\}) > \alpha$, save to $N_2$. \\
$OD(\{$$F_2, F_4$$\}) < \alpha$, remove. \\
$OD(\{$$F_2, F_3, F_4$$\}) < \alpha$, remove. \\
}
%Sort $N_i$ in descending order,
%add subspace with top-k outlying degree in $N_i$ to $O_k$

%%==========================================================================================
\begin{note}
According to the outlying degree scoring function,
we can get the outlying degree of each possible subspace.

So that we can filter out the candidate subspaces.
Next is to sort the candidate subspaces in
descending order and then we pick out the top-k subspaces.

The result is as shown in table $4$.
As a result,
the top-1 group outlying aspect is \{$F_1$\}.
The top-2 group outlying aspects include
trivial feature \{$F_1$\},
and non-trivial subspace \{$F_3, F_4$\}.
\end{note}
%%==========================================================================================

\end{slide}
%%
%%==========================================================================================


%%==========================================================================================
%%
\begin{slide}[toc=,bm=]{Strengths of GOAM Algorithm}
\begin{itemize}
\item
\textcolor{orange}{Reduction of Complexity}

\begin{itemize}
\item
Bottom-up search strategy.

\item
Reduce the size of candidate subspaces.

\end{itemize}

\item
\textcolor{orange}{Efficiency}

\begin{itemize}
\item
Before: $O(2^d)$  \\
Now:    $O(d*n^{2})$

\end{itemize}
\end{itemize}

%%==========================================================================================
\begin{note}
In terms of the strengths of GOAM algorithm.

I would like to talk about two main advantages of this algorithm.
First is the reduction of complexity.
GOAM algorithm utilizes the bottom up search method;
what's more,
it can reduce the size of candidate subspaces.

Second is efficiency.
The previous time complexity is O($2^d$);
however,
current time complexity if only O($d*n^2$).
\end{note}
%%==========================================================================================

\end{slide}
%%
%%==========================================================================================


\section{Evaluation Results}


%%==========================================================================================
%%
\begin{slide}[toc=,bm=]{Evaluation}

\begin{center}
\begin{itemize}

\item
\smallskip
\large
{$Accuracy = \frac{P}{T}$ \\
P: Identified outlying aspects \\

T: Real outlying aspects}

\end{itemize}
\end{center}

%%==========================================================================================
\begin{note}
Before showing the experiment results,
I will introduce the evaluation of the experiment.

We use accuracy formula to make comparisons between GOAM algorithm
and outlying aspect mining method.
In the accuracy formula,
P stands for identified outlying aspects;
and T means the real outlying aspects.
\end{note}
%%==========================================================================================

\end{slide}
%%
%%==========================================================================================


%%==========================================================================================
%%
\begin{slide}{Synthetic Dataset}

\begin{itemize}
\item Synthetic Dataset and Ground Truth
\end{itemize}

\begin{table}
\setlength{\abovecaptionskip}{0pt}
\setlength{\belowcaptionskip}{10pt}
\centering
\caption{Synthetic Dataset and Ground Truth}

\begin{tabular}{p{2.8cm}p{0.9cm}p{0.9cm}p{0.9cm}p{0.9cm}p{0.9cm}p{0.9cm}p{0.9cm}p{0.9cm}}
\hline
  % after \\: \hline or \cline{col1-col2} \cline{col3-col4} ...
  Query group  & $\mathbf{F_1}$ & $\mathbf{F_2}$ & $F_3$ & $\mathbf{F_4}$ & $F_5$ & $F_6$ & $F_7$ & $F_8$\\
\hline
  $i_1$   & \bf{10} & \bf{8}  & 9  & \bf{7}  & 7 & 6 & 6  & 8\\
  $i_2$   & \bf{9}  & \bf{9}  & 7  & \bf{8}  & 9 & 9 & 8  & 9\\
  $i_3$   & \bf{8}  & \bf{10} & 8  & \bf{9}  & 6 & 8 & 7  & 8\\
  $i_4$   & \bf{8}  & \bf{8}  & 6  & \bf{7}  & 8 & 8 & 6  & 7\\
  $i_5$   & \bf{9}  & \bf{9}  & 9  & \bf{7}  & 7 & 7 & 8  & 8\\
  $i_6$   & \bf{8}  & \bf{10} & 8  & \bf{8}  & 6 & 6 & 8  & 7\\
  $i_7$   & \bf{9}  & \bf{9}  & 7  & \bf{9}  & 8 & 8 & 8  & 7\\
  $i_8$   & \bf{10} & \bf{9}  & 10 & \bf{7}  & 7 & 7 & 7  & 7\\
  $i_9$   & \bf{9}  & \bf{10} & 8  & \bf{8}  & 7 & 6 & 7  & 7\\
  $i_{10}$& \bf{9}  & \bf{9}  & 7  & \bf{7}  & 7 & 8 & 8  & 8\\
\hline
\end{tabular}
\end{table}

%%==========================================================================================
\begin{note}
Now,
I am gonna use a synthetic dataset to verify our method.

The dataset we used in our experiment contains $10$ groups,
each group consists of $10$ members,
and each member has $8$ features: $F_1$ to $F_8$.

Table $5$ shows the original data of one group,
and the bold features represent the ground truth,
The ground truth include trivial outlying feature \{$F_1$\},
and non-trivial outlying subspace \{$F_2$, $F_4$\}.
\end{note}
%%==========================================================================================

\end{slide}
%%
%%==========================================================================================


%%==========================================================================================
%%
\begin{slide}[toc=,bm=]{Synthetic Dataset Results}

\begin{table}[tb]
\setlength{\abovecaptionskip}{0pt}
\setlength{\belowcaptionskip}{10pt}
\centering
\caption{The experiment result on synthetic dataset}

\begin{tabular}{ c | c | c | c }
\toprule
  % after \\: \hline or \cline{col1-col2} \cline{col3-col4} ...
  Method     &  Truth Outlying Aspects    & Identified Aspects & Accuracy      \\
\midrule
  GOAM       &  $\{F_1\}$, $\{F_2F_4\}$   &  $\{F_1\}$, $\{F_2F_4\}$    & 100\%    \\

Arithmetic Mean based OAM &  $\{F_1\}$, $\{F_2F_4\}$   &  $\{F_4\}$, $\{F_2\}$    &  0\% \\

Median based OAM &  $\{F_1\}$, $\{F_2F_4\}$   &  $\{F_2\}$, $\{F_4\}$    &           0\% \\
\bottomrule
\end{tabular}
\end{table}

%%==========================================================================================
\begin{note}
From table $6$,
we can see that GOAM method can identify the trivial outlying features
and non-trivial outlying subspaces accurately and
it is obvious from the table that the accuracy of GOAM is the best,
which is 100\%.

This is because the outlying aspects mining method
can't obtain the features of a group and the scoring function
is based on point to point metric.
Therefore,
it is not suitable for group outlying aspects mining.
\end{note}
%%==========================================================================================

\end{slide}
%%
%%==========================================================================================


%%==========================================================================================
%%
\begin{slide}{NBA Dataset}
Data Collection
\begin{description}[type=1]
\item
Source\\
\qquad
\emph{Yahoo Sports} website (\url{http://sports.yahoo.com.cn/nba})

\item
Data

\begin{itemize}
 \item Extract NBA teams' data until March 30, 2018;
 \item 6 divisions;
 \item 12 features (eg: \emph{Point Scored}).
\end{itemize}
\end{description}

%%==========================================================================================
\begin{note}
Next,
I will illustrate it further by applying the GOAM algorithm to the NBA dataset.
The data was collected from Yahoo Sports website.

In our experiment,
a web crawler was deployed to extract data
for all NBA teams until March 30, 2018.
The data includes all teams from the six divisions,
and each player in the team has 12 features,
such as point scored, field goal\dots
\end{note}
%%==========================================================================================

\end{slide}
%%
%%==========================================================================================


%%==========================================================================================
%%
\begin{slide}[toc=,bm=]{NBA Dataset}
The detail features are as follows:

\begin{table}[tb]
\setlength{\abovecaptionskip}{0pt}
\setlength{\belowcaptionskip}{10pt}
\centering
\caption{Collected data of Brooklyn Nets Team}

\begin{tabular}{p{0.9cm}p{0.9cm}p{0.9cm}p{0.9cm}p{0.9cm}p{0.9cm}p{0.9cm}p{0.9cm}p{0.9cm}p{0.9cm}p{0.9cm}p{0.9cm}}
\hline
  Pts & FGA & FG\% & 3FA & 3PT\% & FTA & FT\% & Reb & Ass & To & Stl & Blk \\
\hline
  18   & 12    & 42 &2.00 & 50 & 7.00 & 100& 0& 4& 3& 0& 0 \\
  15.7 & 14.07 & 41 &5.45 & 32 & 3.05 & 75 & 3.98& 5.1& 2.98& 0.69& 0.36\\
  14.5 & 11.1  & 47 &0.82 & 26 & 4.87 & 78 & 6.82& 2.4& 1.74& 0.92& 0.66 \\
  13.5 & 10.8  & 42 &5.37 & 37 & 3.38 & 77 & 6.66& 2& 1.38& 0.83& 0.42 \\
  12.7 & 10.59 & 39 &5.36 & 33 & 3.37 & 82 & 3.24& 6.6& 1.56& 0.89& 0.31 \\
  12.6 & 10.93 & 40 &6.94 & 37 & 1.70 & 84 & 4.27& 1.5& 1.06& 0.61& 0.44 \\
  12.2 & 10.39 & 44 &3.42 & 35 & 2.70 & 72 & 3.79& 4.1& 2.15& 1.12& 0.32 \\
  10.6 & 7.85  & 49 &4.51 & 41 & 1.35 & 83 & 3.34& 1.6& 1.15 & 0.45& 0.24 \\
\hline
\end{tabular}
\end{table}

%%==========================================================================================
\begin{note}
Table $7$ shows part of the collected data.
From this table,
we can see that the feature values are continuous.
\end{note}
%%==========================================================================================

\end{slide}
%%
%%==========================================================================================


%%==========================================================================================
%%
\begin{slide}[toc=,bm=]{NBA Dataset}
\begin{itemize}
\item Data Preprocess
\end{itemize}

\begin{table}
\setlength{\abovecaptionskip}{0pt}
\setlength{\belowcaptionskip}{10pt}
\centering
\caption{The bins that used to discrete data of each feature}

\begin{tabular}{p{2.5cm}p{2cm}p{1.8cm}p{2cm}p{1.8cm}p{1.8cm}p{1.8cm}}
\hline
  Labels & Pts & FGA & FG\% & 3FA & 3PT\% & FTA  \\
\hline
  low &  [0,5]& [0,4] & [0,0.35] & [0,1.0] & [0,0.2] & [0,1.0] \\
  medium& (5,10]& (4,7] & (0.35,0.45] & (1.0,2.5]& (0.2,0.3] & (1.0,1.5] \\
  high &  (10,15] & (7,10] & (0.45,0.5] & (2.5,3.5]& (0.3,0.35]& (1.5,2.5] \\
  very high&(15,$+\infty$]& (10,$+\infty$] & (0.5,1] & (3.5,$+\infty$] & (0.35,1] & (2.5,$+\infty$] \\
\hline
 Labels & FT\% & Reb & Ass & To & Stl & Blk \\
\hline
  low   & [0,0.6] & [0,2.0] & [0,1.0] & [0,0.6] & [0,0.2] & [0,0.25] \\
  medium& (0.6,0.65]& (2,5] & (1,2] & (0.6,0.9] & (0.2,0.5] & (0.25,0.5] \\
  high  & (0.65,0.75] & (5,6] & (2,4] & (0.9,1.7] & (0.6,0.75] & (0.5,0.7] \\
  very high& (0.75,1] & (6,$+\infty$] & (4,$+\infty$] & (1.7,$+\infty$] & (0.75,$+\infty$] & (0.7,$+\infty$]\\
\hline
\end{tabular}
\end{table}

%%==========================================================================================
\begin{note}
For those features with continuous values,
we use the binning method to discretize them,
the results are shown in table $8$.

Once the data was prepared,
we added three teams in the eastern division
and three teams in the western division into the query group,
the other teams together belonged to the contrast groups.
\end{note}
%%==========================================================================================

\end{slide}
%%
%%==========================================================================================


%%==========================================================================================
%%
\begin{slide}[toc=,bm=]{NBA Dataset Results}

\begin{table}[htbp]
\setlength{\abovecaptionskip}{0pt}
\setlength{\belowcaptionskip}{10pt}
\centering
\caption{The identified outlying aspects of groups}

\begin{tabular}{ccc}
\hline
  Teams                   & Trivial Outlying Aspects  & NonTrivial Outlying Aspects    \\
\hline
  Cleveland Cavaliers     & \{3FA\}                   & \{FGA, FT\%\}, \{FGA, FG\%\} \\
  Orlando Magic           & \{Stl\}                   & None                         \\
  Milwaukee Bucks         & \{To\}, \{FTA\}           & \{FGA, FTA\}, \{3FA, FTA\}     \\
  Golden State Warriors   & \{FG\%\}                  & \{FT\%, Blk\}, \{FGA, 3PT\%, FTA\}\\
  Utah Jazz               & \{Blk\}                   & \{3FA, 3PT\%\}                    \\
  New Orleans Pelicans    & \{FT\%\}, \{FTA\}         & \{FTA, Stl\}, \{FTA, To\}          \\
\hline
\end{tabular}
\end{table}

%%==========================================================================================
\begin{note}
We can see the identified group outlying aspects of each team from table $9$.
It is very clear that the GOAM algorithm can identify the
Trivial Outlying Aspects and Non-Trivial Outlying Aspects of each team.
\end{note}
%%==========================================================================================

\end{slide}
%%
%%==========================================================================================


\section{Conclusion}

%%==========================================================================================
%%
\begin{slide}[toc=,bm=]{Conclusion}
\begin{itemize}
\item
\smallskip
Formalize the problem of \emph{Group Outlying Aspects Mining} by
extending outlying aspects mining;

\item
\smallskip
Propose a novel method \textcolor{orange}{GOAM algorithm} to solve the
\emph{Group Outlying Aspects Mining} problem;

\item
\smallskip
Utilize the pruning strategies to reduce time complexity.

\end{itemize}

%%==========================================================================================
\begin{note}
In conclusion,
we firstly formalized the problem of
group outlying aspects mining,

Then proposed a novel method GOAM algorithm to address the problem of
group outlying aspects mining,
and the proposed method use pruning to reduce time complexity
while identifying the suitable set of outlying features for the interested group.

Thank you and any question?
\end{note}
%%==========================================================================================

\end{slide}
%%
%%==========================================================================================


%%==========================================================================================
%
\begin{slide}[toc=,bm=]{Questions?}
\begin{center}
\begin{figure}
    \animategraphics[autoplay, loop, height=0.4\textheight]{5}{figures//gif//question//q_}{1}{30}
\end{figure}
\end{center}
\end{slide}
%%
%%==========================================================================================


%%==========================================================================================
% TODO: Contact Page
\begin{wideslide}[toc=,bm=]{Contact Information}
\centering
\vspace{\stretch{1}}
\twocolumn[
lcolwidth=0.35\linewidth,
rcolwidth=0.65\linewidth
]
{
% \centerline{\includegraphics[scale=.2]{tulip-logo.eps}}
}
{
\vspace{\stretch{1}}
Associate Professor Gang Li\\
School of Information Technology\\
Deakin University, Australia
\begin{description}
 \item[\textcolor{orange}{\faEnvelope}] \href{mailto:gangli@tulip.org.au}
 {\textsc{\footnotesize{gangli@tulip.org.au}}}

 \item[\textcolor{orange}{\faHome}] \href{http://www.tulip.org.au}
 {\textsc{\footnotesize{Team for Universal Learning and Intelligent Processing}}}
\end{description}
}
\vspace{\stretch{1}}
\end{wideslide}

\end{document}

\endinput
